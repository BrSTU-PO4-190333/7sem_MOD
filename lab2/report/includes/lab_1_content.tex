\begin{center}
  \textbf{Отчёт лабораторной работы №\envReportLabNumber}
\end{center}

\textbf{Тема}:
<<\envReportTitle>>

\textbf{Цель}:
изучить основы методов Machine Learning в контексте задачи кластерного анализа (cluster analysis),
приобрести навыки работы с методами Machine Learning в системе STATISTICA StatSoft,
осуществить обработку методами Machine Learning индивидуального набора данных и интерпретацию результатов.

% = = = = = = = = = = = = = = = =

\phantomsection
\addcontentsline{toc}{section}{Подготовка к лабораторной работе}

\begin{center}
  \textbf{Подготовка к лабораторной работе}
\end{center}

Для лабораторной работы нужна программа <<Statistica 10>>.
У меня операционная система Ubuntu 22.04.
Statistica 10 под операционную систему WinXP - Win7.
Чтобы запустить на своей операционной системе программу я буду использовать виртуальную машину.

Чтобы установить Windows XP мне нужен диск (ISO-образ).
Microsoft не поддерживает Windows XP в 2022 году.
С сервером Microsoft не скачать Windows XP.
Буду скачивать ISO-образ с торрент \cite{TorrentWinXP}.
Торрент буду скачивать через open source программу qBitTorrent \cite{downloadQBitTorrent}.

В качестве виртуальной машины буду использовать Oracle Virtual Box \cite{downloadVirtualBox}.
Подключаю ISO образ к виртуальной машине и устанавливаю Windows XP.

Statistica 10 прийдется скачать с торрент трекера \cite{downloadStatistica10}.
Нашел не устанавливаемую версию, а portable (переносимую), которая не засоряет систему.
Запускается *.exe файл и все настройки хранятся в папке "STATISTICA 10 RU" рядом с файлом Ststatistica\_10\_ru\_portable.exe. 

Statistica 10 сразу не запуститься, ибо скажет, что не хватает dll.
Скачаем Visual C++ 2008 \cite{downloadVisualCpp2008}.

Для удобства передачи файлов можно использовать либо через почту и мессенджеры в браузере Firefox 52.9.0esr \cite{downloadFirefox5290esr},
либо подключить общий папки (диск Z).

\begin{center}
  \textbf{Пример использования кластерного анализа STATISTICA в автостраховании}
\end{center}

Я в Libre Office Calc \cite{downloadLibreOffice} создал файл
<<cars\_my.xls>> и заполнил данные как в методичке для того, чтобы попробовать сделать так как в методичке по примеру, а затем сделать свой 5-ый вариант.

\lstinputlisting[name=cars_my.csv (cars_my.xls)]
{../sources/cars_my/cars_my.csv}

Удаляю папку <<Statistica 10 RUS>>.

Запускаю Ststatistica\_10\_ru\_portable.exe.

Главная > Открыть > cars\_my.xls > Открыть \\
> Импортировать выбранные лист в Таблицу данных > Sheet1 > OK \\
> Имена переменных из первой строки>Имена наблюдений из первого столбца > OK

Результат смотри на рисунке~\ref{fig:1_1}.

\begin{figure}[!h]
  \centering

  \includegraphics[height=12cm]
  {inc/cars_my/1.1.PNG}

  \caption{Открыли cars\_my.xls в Statistica 10}

  \label{fig:1_1}
\end{figure}

\begin{center}
  \textbf{Масштаб измерений}
\end{center}

Вид > Заголовки переменных > Отображать тип

1 PRICE Вещественный > Формат отображения > Число > Дес. разряды > 3 > OK

2 age Вещественный > Формат отображения > Число > Дес. разряды > 6 > OK

3 experience Вещественный > Тип > Целый > OK > Да

4 age\_car Вещественный > Тип > Целый > OK > Да

Результат смотри на рисунке~\ref{fig:1_2}.

\begin{figure}[!h]
  \centering

  \includegraphics[height=10cm]
  {inc/cars_my/1.2.PNG}

  \caption{Данные}

  \label{fig:1_2}
\end{figure}

\newpage

Данные > Стандартизировать > Переменные > Выбрать все > OK > OK

Результат смотри на рисунке~\ref{fig:1_3}.

\begin{figure}[!h]
  \centering

  \includegraphics[height=10cm]
  {inc/cars_my/1.3.PNG}

  \caption{Данные}

  \label{fig:1_3}
\end{figure}

Кружок > Сохранить как... > cars\_my\_standartiz.sta > Сохранить

\newpage

\begin{center}
  \textbf{Шаг 1. Иерархическая классификация}
\end{center}

Анализ > Многомерный анализ > Кластерный анализ > Иерархическая классификация > OK \\
> Дополнительно > Переменные > Выбрать все > OK \\
> Объекты > Наблюдения (строки) \\
> Правило объединения > Метод полной связи \\
> Мера близости > Евклидово расстояние \\
> OK > Дополнительно > Вертикальная дендрограмма

Результаты смотри на рисунках~\ref{fig:1_4}, \ref{fig:1_5}, \ref{fig:1_6}.

\begin{figure}[!h]
  \centering
  \begin{minipage}{0.32\textwidth}
    \centering

    \includegraphics[width=0.99\textwidth]
    {inc/cars_my/1.4.PNG}

    \caption{Окно <<Методы кластеризации>>}
    \label{fig:1_4}
  \end{minipage}
  \begin{minipage}{0.32\textwidth}
    \centering

    \includegraphics[width=0.99\textwidth]
    {inc/cars_my/1.5.PNG}

    \caption{Окно <<Кластерный анализ>>}
    \label{fig:1_5}
  \end{minipage}
  \begin{minipage}{0.32\textwidth}
    \centering

    \includegraphics[width=0.99\textwidth]
    {inc/cars_my/1.6.PNG}

    \caption{Окно <<Результаты иерархической классификации>>}
    \label{fig:1_6}
  \end{minipage}
\end{figure}

Результат смотри на рисунке~\ref{fig:1_7}.

\begin{figure}[!h]
  \centering

  \includegraphics[height=11cm]
  {inc/cars_my/1.7.PNG}

  \caption{Дендрограмма для 22 набл.}
  \label{fig:1_7}
\end{figure}

\newpage

\begin{center}
  \textbf{Шаг 2. Кластеризация методом К средних}
\end{center}

Результата иерархической классификации: cars\_my\_standartiz > Отмена > Отмена \\
> Кластеризация методом К средних > OK \\
> Дополнительно \\
> Переменные > Выбрать все > OK \\
> Объекты > Наблюдения (строки) \\
> Число кластеров > 4 \\
> OK > Дополнительно > Дисперсионный анализ

Результаты смотри на рисунках~\ref{fig:2_1}, \ref{fig:2_2}, \ref{fig:2_3}.

\begin{figure}[!h]
  \centering
  \begin{minipage}{0.49\textwidth}
    \centering

    \includegraphics[width=0.99\textwidth]
    {inc/cars_my/2.1.PNG}

    \caption{Окно <<Методы кластеризации>>}
    \label{fig:2_1}
  \end{minipage}
  \begin{minipage}{0.49\textwidth}
    \centering

    \includegraphics[width=0.99\textwidth]
    {inc/cars_my/2.2.PNG}

    \caption{Окно <<Кластерный анализ>>}
    \label{fig:2_2}
  \end{minipage}
\end{figure}

\begin{figure}[!h]
  \centering

  \includegraphics[height=6cm]
  {inc/cars_my/2.3.PNG}

  \caption{Дисперсионный анализ}
  \label{fig:2_3}
\end{figure}

Результаты смотри на рисунке~\ref{fig:2_4}.

\begin{figure}[!h]
  \centering

  \includegraphics[height=3cm]
  {inc/cars_my/2.4.PNG}

  \caption{Дисперсионный анализ}
  \label{fig:2_4}
\end{figure}

\newpage

Результаты метода К средних: cars\_my\_standartiz > Дополнительно > Элементы кластеров и расстояния

Данные первого кластера смотри на рисунке~\ref{fig:2_5}.

Данные второго кластера смотри на рисунке~\ref{fig:2_6}.

Данные третьего кластера смотри на рисунке~\ref{fig:2_7}.

Данные четвертого кластера смотри на рисунке~\ref{fig:2_8}.

\begin{figure}[!h]
  \centering
  \begin{minipage}{0.49\textwidth}
    \centering

    \includegraphics[width=0.99\textwidth]
    {inc/cars_my/2.5.PNG}

    \caption{Первый кластер}
    \label{fig:2_5}
  \end{minipage}
  \begin{minipage}{0.49\textwidth}
    \centering

    \includegraphics[width=0.99\textwidth]
    {inc/cars_my/2.6.PNG}

    \caption{Второй кластер}
    \label{fig:2_6}
  \end{minipage}
\end{figure}

\begin{figure}[!h]
  \centering
  \begin{minipage}{0.49\textwidth}
    \centering

    \includegraphics[width=0.99\textwidth]
    {inc/cars_my/2.7.PNG}

    \caption{Третий кластер}
    \label{fig:2_7}
  \end{minipage}
  \begin{minipage}{0.49\textwidth}
    \centering

    \includegraphics[width=0.99\textwidth]
    {inc/cars_my/2.8.PNG}

    \caption{Четвертый кластер}
    \label{fig:2_8}
  \end{minipage}
\end{figure}

\newpage

\begin{center}
  \textbf{Шаг 3. Кластеризация методом К средних}
\end{center}

Результаты метода К средних: cars\_my\_standartiz > Дополнительно > Сохранить классификацию и расстояние > Выбрать все > OK

Результаты смотри на рисунках~\ref{fig:3_1}, \ref{fig:3_2}, \ref{fig:3_3}, \ref{fig:3_4}.

\begin{figure}[!h]
  \centering
  \begin{minipage}{0.49\textwidth}
    \centering

    \includegraphics[width=0.99\textwidth]
    {inc/cars_my/3.1.PNG}

    \caption{Окно <<Основные статистики и таблицы>>}
    \label{fig:3_1}
  \end{minipage}
  \begin{minipage}{0.49\textwidth}
    \centering

    \includegraphics[width=0.99\textwidth]
    {inc/cars_my/3.2.PNG}

    \caption{Окно <<Внутригрупповые статистики и корреляции>>}
    \label{fig:3_2}
  \end{minipage}
\end{figure}

\begin{figure}[!h]
  \centering
  \begin{minipage}{0.49\textwidth}
    \centering

    \includegraphics[width=0.99\textwidth]
    {inc/cars_my/3.3.PNG}

    \caption{Окно <<Выберите зависимые и группирующие переменные>>}
    \label{fig:3_3}
  \end{minipage}
  \begin{minipage}{0.49\textwidth}
    \centering

    \includegraphics[width=0.99\textwidth]
    {inc/cars_my/3.4.PNG}

    \caption{Окно <<Внутригрупповые статистики и корреляции>>}
    \label{fig:3_4}
  \end{minipage}
\end{figure}

\newpage

Анализ > Основные статистики и таблицы\\
> Группировка и однофакторный ДА > OK\\
> Переменные > 1-PRICE > 6-КЛАСТЕР > OK > OK\\
> Статистики > Минимум и максимум\\
> Описательные > Итоговая таблица средних

Анализ > Основные статистики и таблицы > Новый\\
> Группировка и однофакторный ДА > OK\\
> Переменные > 2-age > 6-КЛАСТЕР > OK > OK\\
> Описательные > Статистики > Минимум и максимум\\
> Итоговая таблица средних

\begin{figure}[!h]
  \centering
  \begin{minipage}{0.49\textwidth}
    \centering

    \includegraphics[width=0.99\textwidth]
    {inc/cars_my/3.5.PNG}

    \caption{PRICE}
    \label{fig:3_5}
  \end{minipage}
  \begin{minipage}{0.49\textwidth}
    \centering

    \includegraphics[width=0.99\textwidth]
    {inc/cars_my/3.6.PNG}

    \caption{age}
    \label{fig:3_6}
  \end{minipage}
\end{figure}

Анализ > Основные статистики и таблицы > Новый\\
> Группировка и однофакторный ДА > OK\\
> Переменные > 3-experience > 6-КЛАСТЕР > OK > OK\\
> Описательные > Статистики > Минимум и максимум\\
> Итоговая таблица средних

Анализ > Основные статистики и таблицы > Новый\\
> Группировка и однофакторный ДА > OK\\
> Переменные > 4-age\_car > 6-КЛАСТЕР > OK > OK\\
> Описательные > Статистики > Минимум и максимум\\
> Итоговая таблица средних

\begin{figure}[!h]
  \centering
  \begin{minipage}{0.49\textwidth}
    \centering

    \includegraphics[width=0.99\textwidth]
    {inc/cars_my/3.7.PNG}

    \caption{experience}
    \label{fig:3_7}
  \end{minipage}
  \begin{minipage}{0.49\textwidth}
    \centering

    \includegraphics[width=0.99\textwidth]
    {inc/cars_my/3.8.PNG}

    \caption{age\_car}
    \label{fig:3_8}
  \end{minipage}
\end{figure}

Анализ > Основные статистики и таблицы > Новый\\
> Группировка и однофакторный ДА > OK\\
> Переменные > (1-PRICE, 2-age, 3-experience, 4-age\_car) > 6-КЛАСТЕР > OK > OK\\
> Быстрый > Графики взаимодействий

Результаты смотри на рисунках~\ref{fig:3_9}, \ref{fig:3_10}, \ref{fig:3_11}, \ref{fig:3_12}, \ref{fig:3_13}, \ref{fig:3_14}.

\begin{figure}[!h]
  \centering
  \begin{minipage}{0.32\textwidth}
    \centering

    \includegraphics[width=0.99\textwidth]
    {inc/cars_my/3.9.PNG}

    \caption{Окно <<Основные статистики и таблицы>>}
    \label{fig:3_9}
  \end{minipage}
  \begin{minipage}{0.32\textwidth}
    \centering

    \includegraphics[width=0.99\textwidth]
    {inc/cars_my/3.10.PNG}

    \caption{Окно <<Выберите зависимые и группирующие переменные>>}
    \label{fig:3_10}
  \end{minipage}
  \begin{minipage}{0.32\textwidth}
    \centering

    \includegraphics[width=0.99\textwidth]
    {inc/cars_my/3.11.PNG}

    \caption{Окно <<Внутригрупповые статистики и корреляции>>}
    \label{fig:3_11}
  \end{minipage}
\end{figure}

\begin{figure}[!h]
  \centering
  \begin{minipage}{0.49\textwidth}
    \centering

    \includegraphics[width=0.99\textwidth]
    {inc/cars_my/3.12.PNG}

    \caption{Окно <<Внутригрупповые статистики и корреляции>>}
    \label{fig:3_12}
  \end{minipage}
  \begin{minipage}{0.49\textwidth}
    \centering

    \includegraphics[width=0.99\textwidth]
    {inc/cars_my/3.13.PNG}

    \caption{Окно <<Переменные для графика взаимодействий>>}
    \label{fig:3_13}
  \end{minipage}
\end{figure}

\begin{figure}[!h]
  \centering

  \includegraphics[height=8.2cm]
  {inc/cars_my/3.14.PNG}

  \caption{График средних и дов. интервалов (95.00\%)}
  \label{fig:3_14}
\end{figure}

% Результат действий смотри на рисунке~\ref{fig:1}.

% \begin{figure}[!h]
%   \centering

%   \includegraphics[height=6cm]
%   {inc/Series_G/1.PNG}

%   \caption{Общие сведения}

%   \label{fig:1}
% \end{figure}

% \begin{center}
%   \textbf{Определение анализа}
% \end{center}

% Анализ > Углубленные методы > Временные ряды и прогнозирование\\
% > Переменные > ОК\\
% > Методы > АРПСС и автокорреляционные функции > Методы

% Результат действий смотри на рисунке~\ref{fig:2}.

% \begin{figure}[!h]
%   \centering

%   \includegraphics[height=6cm]
%   {inc/Series_G/2.PNG}

%   \caption{Определение анализа}

%   \label{fig:2}
% \end{figure}

% \begin{center}
%   \textbf{Фаза идентификации}
% \end{center}

% Анализ > Углубленные методы > Временные ряды и прогнозирование > Новый\\
% > Переменные > ОК\\
% > Методы > АРПСС и автокорреляционные функции > Методы \\
% > Дополнительно > Другие преобразования и графики\\
% > Графики\\
% > Задать масштаб по оси Х (мин, знач, шаг) > 1 > 12\\
% > Пометить точки по оси Х > Именами наблюдений\\
% > График (рядом с кнопкой Просмотр выдел. Переменной)

% Результат действий смотри на рисунке~\ref{fig:3}.

% \begin{figure}[!h]
%   \centering

%   \includegraphics[height=6cm]
%   {inc/Series_G/3.PNG}

%   \caption{Фаза идентификации}

%   \label{fig:3}
% \end{figure}

% \begin{center}
%   \textbf{Мультипликативная сезонность}
% \end{center}

% % Из графика ряда также видно, что амплитуда сезонных изменений со временем увеличивается
% % (т.е. есть свидетельства мультипликативной сезонности), что может смещать значения автокорреляций.
% % Для стабилизации этой изменчивости будет выполнено преобразование данных в натуральный логарифм.

% \begin{center}
%   \textbf{Логарифмическое преобразование}
% \end{center}

% Преобразование переменных Series\_G > x=f(x)\\
% > Преобразования > Натуральный логарифм ( x=ln(x) )\\
% > OK (Преобразовать выделенную переменную)

% Результат действий смотри на рисунке~\ref{fig:4}.

% \begin{figure}[!h]
%   \centering

%   \includegraphics[height=7cm]
%   {inc/Series_G/4.PNG}

%   \caption{Логарифмическое преобразование}

%   \label{fig:4}
% \end{figure}

% \begin{center}
%   \textbf{Автокорреляции}
% \end{center}

% Преобразование переменных Series\_G > Автокорреляция\\
% > Число лагов > 25\\
% > Автокорреляция

% Результат действий смотри на рисунке~\ref{fig:5}.

% \begin{figure}[!h]
%   \centering

%   \includegraphics[height=7cm]
%   {inc/Series_G/5.PNG}

%   \caption{Логарифмическое преобразование}

%   \label{fig:5}
% \end{figure}

% % = = = = = = = = = = = = = = = = = = = = = = = = = = = = = = = =

% \newpage

% \begin{center}
%   \textbf{Разность}
% \end{center}

% Преобразование переменных Series\_G\\
% > Разность, сумма\\
% > Преобразования  > Разность ( x = x - x(лаг) )\\
% > ОК (Преобразовать выделенную переменную)

% Результат действий смотри на рисунке~\ref{fig:6}.

% \begin{figure}[!h]
%   \centering

%   \includegraphics[height=8cm]
%   {inc/Series_G/6.PNG}

%   \caption{Разность}

%   \label{fig:6}
% \end{figure}

% Преобразование переменных Series\_G\\
% > Автокорреляции\\
% > Автокорреляции и кросскорреляции > Автокорреляции

% Результат действий смотри на рисунке~\ref{fig:7}.

% \begin{figure}[!h]
%   \centering

%   \includegraphics[height=8cm]
%   {inc/Series_G/7.PNG}

%   \caption{Разность}

%   \label{fig:7}
% \end{figure}

% % = = = = = = = = = = = = = = = = = = = = = = = = = = = = = = = =

% \newpage

% \begin{center}
%   \textbf{Сезонность}
% \end{center}

% \begin{center}
%   \textbf{Взятие сезонной разности}
% \end{center}

% Преобразование переменных Series\_G\\
% > Разность, сумма\\
% > Преобразования  > Разность ( x = x - x(лаг) ) > Лаг=12\\
% > ОК (Преобразовать выделенную переменную)

% Результат действий смотри на рисунке~\ref{fig:8}.

% \begin{figure}[!h]
%   \centering

%   \includegraphics[height=7cm]
%   {inc/Series_G/8.PNG}

%   \caption{Взятие сезонной разности}

%   \label{fig:8}
% \end{figure}

% Преобразование переменных Series\_G\\
% > Графики\\
% > Графики после каждого преобразования (снять галочку)\\
% > Автокорреляции\\
% > Автокорреляции

% Результат действий смотри на рисунке~\ref{fig:9}.

% \begin{figure}[!h]
%   \centering

%   \includegraphics[height=7cm]
%   {inc/Series_G/9.PNG}

%   \caption{Взятие сезонной разности}

%   \label{fig:9}
% \end{figure}

% % = = = = = = = = = = = = = = = = = = = = = = = = = = = = = = = =

% \newpage

% Преобразование переменных Series\_G\\
% > Автокорреляции\\
% > Частые Автокорреляции

% Результат действий смотри на рисунке~\ref{fig:10}.

% \begin{figure}[!h]
%   \centering

%   \includegraphics[height=6cm]
%   {inc/Series_G/10.PNG}

%   \caption{Взятие сезонной разности}

%   \label{fig:10}
% \end{figure}

% % \begin{center}
% %   \textbf{Параметры, подлежащие оценке}
% % \end{center}

% % Коррелограмма выглядит хорошо, и теперь ряд готов для ARIMA.
% % Основываясь на изучении природы ряда (т.е. на этапе идентификации ARIMA),
% % можно прийти к выводу, что сезонная АРПСС (с лагом 12)
% % и несезонная модель (с лагом 1) достаточно хорошо подходят к преобразованному ряду.
% % Будут оцениваться два параметра скользящего среднего модели АРПСС:
% % один сезонный (Qs) и один несезонный (q).
% % Параметры авторегрессии отсутствуют в модели.

% \begin{center}
%   \textbf{Диалог спецификаций ARIMA}
% \end{center}

% Преобразование переменных Series\_G\\
% > Отмена\\
% > L SERIES\_G Montly passenger totals (in 1000's)\\
% > Методы\\
% > Преобразовать переменную (ряд) перед анализом\\
% > Натур. логарифм (установить флажок)\\
% > Разность (установить флажок)\\
% > 1. Лаг=1 > Порядок разности=1\\
% > 2. Лаг=12 > Порядок разности=1\\
% > q - скольз. средн.=1 > Q - Сезонных=1

% Результат действий смотри на рисунке~\ref{fig:11}.

% \begin{figure}[!h]
%   \centering

%   \includegraphics[height=6cm]
%   {inc/Series_G/11.PNG}

%   \caption{Диалог спецификаций ARIMA}

%   \label{fig:11}
% \end{figure}

% % = = = = = = = = = = = = = = = = = = = = = = = = = = = = = = = =

% \begin{center}
%   \textbf{Оценивание параметров}
% \end{center}

% > ОК (Начать оценивание параметров)

% Результат действий смотри на рисунке~\ref{fig:12}.

% \begin{figure}[!h]
%   \centering

%   \includegraphics[height=9cm]
%   {inc/Series_G/12.PNG}

%   \caption{Оценивание параметров}

%   \label{fig:12}
% \end{figure}

% \begin{center}
%   \textbf{Вывод ARIMA}
% \end{center}

% > ОК

% Результат действий смотри на рисунке~\ref{fig:13}.

% \begin{figure}[!h]
%   \centering

%   \includegraphics[height=8cm]
%   {inc/Series_G/13.PNG}

%   \caption{Вывод ARIMA}

%   \label{fig:13}
% \end{figure}

% % = = = = = = = = = = = = = = = = = = = = = = = = = = = = = = = =

% \newpage

% \begin{center}
%   \textbf{Параметры прогноза}
% \end{center}

% Результаты АРПСС: Series\_G
% > Дополнительно > Прогноз

% Результат действий смотри на рисунке~\ref{fig:14}.

% \begin{figure}[!h]
%   \centering

%   \includegraphics[height=8cm]
%   {inc/Series_G/14.PNG}

%   \caption{Вывод ARIMA}

%   \label{fig:14}
% \end{figure}

% \begin{center}
%   \textbf{График прогнозов}
% \end{center}

% Результаты АРПСС: Series\_G
% > Дополнительно > График ряда и прогнозов

% Результат действий смотри на рисунке~\ref{fig:15}.

% \begin{figure}[!h]
%   \centering

%   \includegraphics[height=8cm]
%   {inc/Series_G/15.PNG}

%   \caption{График прогнозов}

%   \label{fig:15}
% \end{figure}

% % = = = = = = = = = = = = = = = = = = = = = = = = = = = = = = = =

% \newpage

% Результаты АРПСС: Series\_G\\
% > Дополнительно
% > Число набл.=12
% > Начать с=133\\
% > График ряда и прогнозов

% Результат действий смотри на рисунке~\ref{fig:16}.

% \begin{figure}[!h]
%   \centering

%   \includegraphics[height=8cm]
%   {inc/Series_G/16.PNG}

%   \caption{График прогнозов}

%   \label{fig:16}
% \end{figure}

% \begin{center}
%   \textbf{Графики нормальной вероятности}
% \end{center}

% Результаты АРПСС: Series\_G\\
% > Распределение остатков
% > Нормальный график

% Результат действий смотри на рисунке~\ref{fig:17}.

% \begin{figure}[!h]
%   \centering

%   \includegraphics[height=8cm]
%   {inc/Series_G/17.PNG}

%   \caption{Графики нормальной вероятности}

%   \label{fig:17}
% \end{figure}

% % = = = = = = = = = = = = = = = = = = = = = = = = = = = = = = = =

% \newpage

% Результаты АРПСС: Series\_G\\
% > Распределение остатков
% > Нормальный график без тренда

% Результат действий смотри на рисунке~\ref{fig:18}.

% \begin{figure}[!h]
%   \centering

%   \includegraphics[height=9cm]
%   {inc/Series_G/18.PNG}

%   \caption{Нормальный график без тренда}

%   \label{fig:18}
% \end{figure}

% Результаты АРПСС: Series\_G\\
% > Распределение остатков
% > Гистограмма

% Результат действий смотри на рисунке~\ref{fig:19}.

% \begin{figure}[!h]
%   \centering

%   \includegraphics[height=9cm]
%   {inc/Series_G/19.PNG}

%   \caption{Гистограмма}

%   \label{fig:19}
% \end{figure}

% \newpage

% \begin{center}
%   \textbf{Автокорреляция остатков}
% \end{center}

% Результаты АРПСС: Series\_G\\
% > Автокорреляции
% > Автокорреляции остатков
% > Автокорреляции

% Результат действий смотри на рисунке~\ref{fig:20}.

% \begin{figure}[!h]
%   \centering

%   \includegraphics[height=8cm]
%   {inc/Series_G/20.PNG}

%   \caption{Автокорреляция остатков}

%   \label{fig:20}
% \end{figure}

% \begin{center}
%   \textbf{Дальнейшие анализы}
% \end{center}

% Результаты АРПСС: Series\_G > Отмена\\
% > Дополнительно > Метод оценивания > Точный (Меларда)\\

% Результат действий смотри на рисунке~\ref{fig:21}.

% \begin{figure}[!h]
%   \centering

%   \includegraphics[height=8cm]
%   {inc/Series_G/21.PNG}

%   \caption{Дальнейшие анализы}

%   \label{fig:21}
% \end{figure}

% \newpage

% > Прогноз > График двух списков перем. в разных масшт.\\
% > SERIES\_G: +прогнозы, Модель: (0,1.1)(0,1,1);\\
% > SERIES\_G: АРПСС (0,1,1)(0,1,1) остатки;\\

% Результат действий смотри на рисунке~\ref{fig:22}.

% \begin{figure}[!h]
%   \centering

%   \includegraphics[height=8cm]
%   {inc/Series_G/22.PNG}

%   \caption{Дальнейшие анализы}

%   \label{fig:22}
% \end{figure}

% > OK

% Результат действий смотри на рисунке~\ref{fig:23}.

% \begin{figure}[!h]
%   \centering

%   \includegraphics[height=10cm]
%   {inc/Series_G/23.PNG}

%   \caption{Дальнейшие анализы}

%   \label{fig:23}
% \end{figure}

% % = = = = = = = = = = = = = = = = = = = = = = = = = = = = = = = =
% % = = = = = = = = = = = = = = = = = = = = = = = = = = = = = = = =
% % = = = = = = = = = = = = = = = = = = = = = = = = = = = = = = = =

% \newpage

% \phantomsection
% \addcontentsline{toc}{section}{Вариант 2}

% \begin{center}
%   \textbf{Вариант 2}
% \end{center}

% % \phantomsection
% % \addcontentsline{toc}{section}{Вариант 2}

% \begin{center}
%   \textbf{(Canada Gas Production)}
% \end{center}

% \begin{center}
%   \textbf{Общие сведения}
% \end{center}

% Главная > Открыть > "Canada Gas Production.xlsx"\\
% > Импортировать выбранный лист в Таблицу данных
% > Sheet1\\
% > Имена переменных из первой строки\\
% > Имена наблюдений из первого столбца\\
% > Импорт формата ячеек\\
% > ОК

% Результат действий смотри на рисунке~\ref{fig:2_1}.

% \begin{figure}[!h]
%   \centering

%   \includegraphics[height=16cm]
%   {inc/Canada_Gas_Production/1.PNG}

%   \caption{Общие сведения}

%   \label{fig:2_1}
% \end{figure}

% \newpage

% \begin{center}
%   \textbf{Определение анализа}
% \end{center}

% Анализ > Углубленные методы > Временные ряды и прогнозирование\\
% > Переменные > ОК\\
% > Методы > АРПСС и автокорреляционные функции > Методы

% Результат действий смотри на рисунке~\ref{fig:2_2}.

% \begin{figure}[!h]
%   \centering

%   \includegraphics[height=6cm]
%   {inc/Canada_Gas_Production/2.PNG}

%   \caption{Определение анализа}

%   \label{fig:2_2}
% \end{figure}

% \begin{center}
%   \textbf{Фаза идентификации}
% \end{center}

% Анализ > Углубленные методы > Временные ряды и прогнозирование > Новый\\
% > Переменные > ОК\\
% > Методы > АРПСС и автокорреляционные функции > Методы \\
% > Дополнительно > Другие преобразования и графики\\
% > Графики\\
% > Задать масштаб по оси Х (мин, знач, шаг) > 1 > 12\\
% > Пометить точки по оси Х > Именами наблюдений\\
% > График (рядом с кнопкой Просмотр выдел. Переменной)

% Результат действий смотри на рисунке~\ref{fig:2_3}.

% \begin{figure}[!h]
%   \centering

%   \includegraphics[height=6cm]
%   {inc/Canada_Gas_Production/3.PNG}

%   \caption{Фаза идентификации}

%   \label{fig:2_3}
% \end{figure}

% \begin{center}
%   \textbf{Мультипликативная сезонность}
% \end{center}

% % Из графика ряда также видно, что амплитуда сезонных изменений со временем увеличивается
% % (т.е. есть свидетельства мультипликативной сезонности), что может смещать значения автокорреляций.
% % Для стабилизации этой изменчивости будет выполнено преобразование данных в натуральный логарифм.

% \begin{center}
%   \textbf{Логарифмическое преобразование}
% \end{center}

% Преобразование переменных Series\_G > x=f(x)\\
% > Преобразования > Натуральный логарифм ( x=ln(x) )\\
% > OK (Преобразовать выделенную переменную)

% Результат действий смотри на рисунке~\ref{fig:2_4}.

% \begin{figure}[!h]
%   \centering

%   \includegraphics[height=7cm]
%   {inc/Canada_Gas_Production/4.PNG}

%   \caption{Логарифмическое преобразование}

%   \label{fig:2_4}
% \end{figure}

% \begin{center}
%   \textbf{Автокорреляции}
% \end{center}

% Преобразование переменных Series\_G > Автокорреляция\\
% > Число лагов > 25\\
% > Автокорреляция

% Результат действий смотри на рисунке~\ref{fig:2_5}.

% \begin{figure}[!h]
%   \centering

%   \includegraphics[height=7cm]
%   {inc/Canada_Gas_Production/5.PNG}

%   \caption{Логарифмическое преобразование}

%   \label{fig:2_5}
% \end{figure}

% % = = = = = = = = = = = = = = = = = = = = = = = = = = = = = = = =

% \newpage

% \begin{center}
%   \textbf{Разность}
% \end{center}

% Преобразование переменных: Canada\_Gas\_Production\\
% > Разность, сумма\\
% > Преобразования  > Разность ( x = x - x(лаг) )\\
% > ОК (Преобразовать выделенную переменную)

% Результат действий смотри на рисунке~\ref{fig:2_6}.

% \begin{figure}[!h]
%   \centering

%   \includegraphics[height=8cm]
%   {inc/Canada_Gas_Production/6.PNG}

%   \caption{Разность}

%   \label{fig:2_6}
% \end{figure}

% Преобразование переменных: Canada\_Gas\_Production\\
% > Автокорреляции\\
% > Автокорреляции и кросскорреляции > Автокорреляции

% Результат действий смотри на рисунке~\ref{fig:2_7}.

% \begin{figure}[!h]
%   \centering

%   \includegraphics[height=8cm]
%   {inc/Canada_Gas_Production/7.PNG}

%   \caption{Разность}

%   \label{fig:2_7}
% \end{figure}

% % = = = = = = = = = = = = = = = = = = = = = = = = = = = = = = = =

% \newpage

% \begin{center}
%   \textbf{Сезонность}
% \end{center}

% \begin{center}
%   \textbf{Взятие сезонной разности}
% \end{center}

% Преобразование переменных: Canada\_Gas\_Production\\
% > Разность, сумма\\
% > Преобразования  > Разность ( x = x - x(лаг) ) > Лаг=12\\
% > ОК (Преобразовать выделенную переменную)

% Результат действий смотри на рисунке~\ref{fig:2_8}.

% \begin{figure}[!h]
%   \centering

%   \includegraphics[height=7cm]
%   {inc/Canada_Gas_Production/8.PNG}

%   \caption{Взятие сезонной разности}

%   \label{fig:2_8}
% \end{figure}

% Преобразование переменных: Canada\_Gas\_Production\\
% > Графики\\
% > Графики после каждого преобразования (снять галочку)\\
% > Автокорреляции\\
% > Автокорреляции

% Результат действий смотри на рисунке~\ref{fig:2_9}.

% \begin{figure}[!h]
%   \centering

%   \includegraphics[height=7cm]
%   {inc/Canada_Gas_Production/9.PNG}

%   \caption{Взятие сезонной разности}

%   \label{fig:2_9}
% \end{figure}

% % = = = = = = = = = = = = = = = = = = = = = = = = = = = = = = = =

% \newpage

% Преобразование переменных: Canada\_Gas\_Production\\
% > Автокорреляции\\
% > Частые Автокорреляции

% Результат действий смотри на рисунке~\ref{fig:2_10}.

% \begin{figure}[!h]
%   \centering

%   \includegraphics[height=6cm]
%   {inc/Canada_Gas_Production/10.PNG}

%   \caption{Взятие сезонной разности}

%   \label{fig:2_10}
% \end{figure}

% % \begin{center}
% %   \textbf{Параметры, подлежащие оценке}
% % \end{center}

% % Коррелограмма выглядит хорошо, и теперь ряд готов для ARIMA.
% % Основываясь на изучении природы ряда (т.е. на этапе идентификации ARIMA),
% % можно прийти к выводу, что сезонная АРПСС (с лагом 12)
% % и несезонная модель (с лагом 1) достаточно хорошо подходят к преобразованному ряду.
% % Будут оцениваться два параметра скользящего среднего модели АРПСС:
% % один сезонный (Qs) и один несезонный (q).
% % Параметры авторегрессии отсутствуют в модели.

% \begin{center}
%   \textbf{Диалог спецификаций ARIMA}
% \end{center}

% Преобразование переменных: Canada\_Gas\_Production\\
% > Отмена\\
% > L SERIES\_G Montly passenger totals (in 1000's)\\
% > Методы\\
% > Преобразовать переменную (ряд) перед анализом\\
% > Натур. логарифм (установить флажок)\\
% > Разность (установить флажок)\\
% > 1. Лаг=1 > Порядок разности=1\\
% > 2. Лаг=12 > Порядок разности=1\\
% > q - скольз. средн.=1 > Q - Сезонных=1

% Результат действий смотри на рисунке~\ref{fig:2_11}.

% \begin{figure}[!h]
%   \centering

%   \includegraphics[height=6cm]
%   {inc/Canada_Gas_Production/11.PNG}

%   \caption{Диалог спецификаций ARIMA}

%   \label{fig:2_11}
% \end{figure}

% % = = = = = = = = = = = = = = = = = = = = = = = = = = = = = = = =

% \begin{center}
%   \textbf{Оценивание параметров}
% \end{center}

% > ОК (Начать оценивание параметров)

% Результат действий смотри на рисунке~\ref{fig:2_12}.

% \begin{figure}[!h]
%   \centering

%   \includegraphics[height=9cm]
%   {inc/Canada_Gas_Production/12.PNG}

%   \caption{Оценивание параметров}

%   \label{fig:2_12}
% \end{figure}

% \begin{center}
%   \textbf{Вывод ARIMA}
% \end{center}

% > ОК

% Результат действий смотри на рисунке~\ref{fig:2_13}.

% \begin{figure}[!h]
%   \centering

%   \includegraphics[height=8cm]
%   {inc/Canada_Gas_Production/13.PNG}

%   \caption{Вывод ARIMA}

%   \label{fig:2_13}
% \end{figure}

% % = = = = = = = = = = = = = = = = = = = = = = = = = = = = = = = =

% \newpage

% \begin{center}
%   \textbf{Параметры прогноза}
% \end{center}

% Результаты АРПСС: Canada Gas Production
% > Дополнительно > Прогноз

% Результат действий смотри на рисунке~\ref{fig:2_14}.

% \begin{figure}[!h]
%   \centering

%   \includegraphics[height=8cm]
%   {inc/Canada_Gas_Production/14.PNG}

%   \caption{Вывод ARIMA}

%   \label{fig:2_14}
% \end{figure}

% \begin{center}
%   \textbf{График прогнозов}
% \end{center}

% Результаты АРПСС: Canada Gas Production\\
% > Дополнительно > График ряда и прогнозов

% Результат действий смотри на рисунке~\ref{fig:2_15}.

% \begin{figure}[!h]
%   \centering

%   \includegraphics[height=8cm]
%   {inc/Canada_Gas_Production/15.PNG}

%   \caption{График прогнозов}

%   \label{fig:2_15}
% \end{figure}

% % = = = = = = = = = = = = = = = = = = = = = = = = = = = = = = = =

% \newpage

% Результаты АРПСС: Canada Gas Production\\
% > Дополнительно
% > Число набл.=12
% > Начать с=242 (253-12+1)\\
% > График ряда и прогнозов

% Результат действий смотри на рисунке~\ref{fig:2_16}.

% \begin{figure}[!h]
%   \centering

%   \includegraphics[height=8cm]
%   {inc/Canada_Gas_Production/16.PNG}

%   \caption{График прогнозов}

%   \label{fig:2_16}
% \end{figure}

% \begin{center}
%   \textbf{Графики нормальной вероятности}
% \end{center}

% Результаты АРПСС: Canada Gas Production\\
% > Распределение остатков
% > Нормальный график

% Результат действий смотри на рисунке~\ref{fig:2_17}.

% \begin{figure}[!h]
%   \centering

%   \includegraphics[height=8cm]
%   {inc/Canada_Gas_Production/17.PNG}

%   \caption{Графики нормальной вероятности}

%   \label{fig:2_17}
% \end{figure}

% % = = = = = = = = = = = = = = = = = = = = = = = = = = = = = = = =

% \newpage

% Результаты АРПСС: Canada Gas Production\\
% > Распределение остатков
% > Нормальный график без тренда

% Результат действий смотри на рисунке~\ref{fig:2_18}.

% \begin{figure}[!h]
%   \centering

%   \includegraphics[height=9cm]
%   {inc/Canada_Gas_Production/18.PNG}

%   \caption{Нормальный график без тренда}

%   \label{fig:2_18}
% \end{figure}

% Результаты АРПСС: Canada Gas Production\\
% > Распределение остатков
% > Гистограмма

% Результат действий смотри на рисунке~\ref{fig:2_19}.

% \begin{figure}[!h]
%   \centering

%   \includegraphics[height=9cm]
%   {inc/Canada_Gas_Production/19.PNG}

%   \caption{Гистограмма}

%   \label{fig:2_19}
% \end{figure}

% \newpage

% \begin{center}
%   \textbf{Автокорреляция остатков}
% \end{center}

% Результаты АРПСС: Canada Gas Production\\
% > Автокорреляции
% > Автокорреляции остатков
% > Автокорреляции

% Результат действий смотри на рисунке~\ref{fig:2_20}.

% \begin{figure}[!h]
%   \centering

%   \includegraphics[height=8cm]
%   {inc/Canada_Gas_Production/20.PNG}

%   \caption{Автокорреляция остатков}

%   \label{fig:2_20}
% \end{figure}

% \begin{center}
%   \textbf{Дальнейшие анализы}
% \end{center}

% Результаты АРПСС: Canada Gas Production > Отмена\\
% > Дополнительно > Метод оценивания > Точный (Меларда)\\

% Результат действий смотри на рисунке~\ref{fig:2_21}.

% \begin{figure}[!h]
%   \centering

%   \includegraphics[height=8cm]
%   {inc/Canada_Gas_Production/21.PNG}

%   \caption{Дальнейшие анализы}

%   \label{fig:2_21}
% \end{figure}

% \newpage

% > Прогноз > График двух списков перем. в разных масшт.\\
% > SERIES\_G: +прогнозы, Модель: (0,1.1)(0,1,1);\\
% > SERIES\_G: АРПСС (0,1,1)(0,1,1) остатки;\\

% Результат действий смотри на рисунке~\ref{fig:2_22}.

% \begin{figure}[!h]
%   \centering

%   \includegraphics[height=8cm]
%   {inc/Canada_Gas_Production/22.PNG}

%   \caption{Дальнейшие анализы}

%   \label{fig:2_22}
% \end{figure}

% > OK

% Результат действий смотри на рисунке~\ref{fig:2_23}.

% \begin{figure}[!h]
%   \centering

%   \includegraphics[height=10cm]
%   {inc/Canada_Gas_Production/23.PNG}

%   \caption{Дальнейшие анализы}

%   \label{fig:2_23}
% \end{figure}
